%!TEX root=../document.tex

\section{Vorträge}
\label{sec:Ergebnisse}
\subsection{Das Gründungs 1 x 1, Saal 1 13:00}
\subsubsection{Inhalt}
	\textbf{Vortragende:} Petra Haslinger - Gründungsexpertin der Wirtschaftskammer Wien.
	
	\textbf{Von der Idee zur Gründung}
	\begin{itemize}
		\item Womit (Produkt, Dienstleistung) möchte ich mich selbstständig machen? Brauche ich dafür eine Gewerbeberechtigung? Wenn ja welche?
		\item Welche Vorraussetzungen muss ich dafür erfüllen und wie komme ich zur Gewerbeberechtigung?
		\item Brauche ich einen Businessplan und was muss in Bezug auf den Unternehmensstandort oder die Rechtsform beachten?
		\item Und am wichtigsten: An wen kann ich mich wenden, wenn ich Fragen zur Unternehmensgründung habe?
	\end{itemize}
\subsubsection{Motivation}

\subsection{FinanzOnline, Steuerrechner, Saal 1 09:00}
\textbf{Vortragender:} Rudolf Weninger - Experte des BM für Finanzen

Hier erfahren Sie Wissenswertes über den richtigen Umgang mit FinanzOnline, dem direkten Draht zum Finanzamt. FinanzOnline erspart Unternehmerinnen und Unternehmern den Weg zum Finanzamt, egal, ob für die Umsatzsteuer-Voranmeldung, die Steuererklärungen, die Bezahlung oder Rückzahlung von Steuern und viele andere Anliegen. Außerdem werden weitere attraktive Angebote auf der BMF-Homepage präsentiert, wie zum Beispiel Steuerrechner, aber auch Informationen zum Steuerrecht.

\subsubsection{Motivation}
\newpage
\subsection{Segel setzen - auf zu neuen Ufern, Saal 2 10:00}

\textbf{Vortragende:}
\begin{itemize}
	\item \textbf{Walter Ruck} - Präsident der Wirtschaftskammer Wien
	\item \textbf{Gerhard Hirczi} - Geschäftsführer der Wirtschaftsagentur Wien
	\item \textbf{Jürgen Tarbauer} - Vorsitzender der Jungen Wirtschaft Wien
\end{itemize}

Ex-Profi Fußballer Peter Alexander Hackmair gehört zu jenen Menschen, die sich für einen Kurswechsel entschieden haben, um sich selbst treu zu bleiben. Nach seinem steilen Aufstieg im Profi-Fußball beendete er nach mehreren Verletzungen mit nur 25 Jahren seine Karriere. Dieser Schritt war nicht das Ende seines Traums, sondern vielmehr der Beginn eines neuen. Am Jungunternehmertag erzählt der heute 29-jährige Buchautor seine Geschichte vom Auf- und Abstieg, wie man sich selbst treu bleibt und was es heißt, seine Träume neu zu gestalten.

Ilse Dippmann, Unternehmerin, Leistungssportlerin und Visionärin Initiatorin und Organisatorin des Österreichi­schen Frauenlaufs wird ebenfalls als Keynote beim Eröffnungstalk mit dabei sein. Die 30fache (!) Marathonläuferin spricht über Parallelen von Unternehmertum und Spitzensport und wird uns erzählen, wie man mit Einsatz und Leidenschaft Menschen in Bewegung bringt.

Kurt Wachter, seit 1999 Projektkoordinator der Initiative fairplay des VIDC- Wiener Institut für internationalen Dialog und Zusammenarbeit, erzählt über die integrative Kraft des Fußballs, über Fairness und wird die neueste Initiative ''Playtogethernow'' vorstellen.
\subsubsection{Motivation}

\subsection{Das Zentrum für erfolgreiche Unternehmen, Saal 6 11:00}

\textbf{Vortragende:} Silvia Fleischhacker  - Stv. Leitung des Forum [EPU KMU]


Das FORUM [EPU KMU] bietet drei konkrete Angebote für Wiener UnternehmerInnen:
\begin{itemize}
	\item Besprechungsräume mit moderner Infrastruktur
	\item Workshops zur Unternehmensführung und persönlichen Entwicklung
	\item Workshops zur Unternehmensführung und persönlichen Entwicklung
	Netzwerkveranstaltungen 
\end{itemize}

Was bietet dieses FORUM im Detail, wie kannst Du davon profitieren und welche Vorteile bringt es, diese Ressourcen von Anfang an zu nutzen?

\subsection{\dfrac{Zähler}{Nenner}}








%!TEX root=../document.tex

\section{Arbeitsrecht}
	\subsection{Aufgabenstellung}
		ÜAWS Recht S. 115ff, Ü3.5.-3.18, W3.6.-3.16.
	\subsection{Ü3.5}
		\textbf{Muss ein Dienstvertrag schriftlich abgeschlossen werden?}
		Für Arbeitsverträge gibt es keine Formvorschriften, sie können daher mündlich, schriftlich oder durch schlüssige Handlung abgeschlossen werden. Wurde der Vertrag nicht schriftlich geschlossen, dann ist der Dienstgeber allerdings verpflichtet, einen Dienstzettel auszustellen.
	
	\subsection{Ü3.6}
		\textbf{Darf sich ein Arbeitnehmer durch jemand anderen vertreten lassen?}
		Der Arbeitnehmer ist zur persönlichen Arbeitsleistung verpflichtet.
		
	\subsection{Ü3.7}
		\textbf{Schreiben Sie zu jeder der folgenden Aussagen, ob Sie richtig oder falsch ist, und begründen Sie Ihre Antwort:}
		\begin{table}[!h]
			\centering
			\caption{Ü3.7}
			\label{my-label}
			\begin{tabular}{|l|l|l|}
				\hline
				\begin{tabular}[c]{@{}l@{}}Es dürfen ohne Zustimmung des Arbeitsinspektors,\\ maximal 5 Überstunden je Woche geleistet werden.\end{tabular}         & Falsch  & 10 Stunden                                                                                                                \\ \hline
				\begin{tabular}[c]{@{}l@{}}Die gesetzliche Normalarbeitszeit von 40 Stunden darf\\ vertraglich verlängert, aber nicht verkürzt werden.\end{tabular} & Falsch  & verkürzt, aber nicht verlängert                                                                                           \\ \hline
				\begin{tabular}[c]{@{}l@{}}Arbeitnehmer sind verpflichtet, Überstunden zu\\ leisten, wenn es der bzw. die Vorgesetzte verlangt.\end{tabular}        & Falsch  & \begin{tabular}[c]{@{}l@{}}auch die Leistung von Überstunden ist zu\\ vereinbaren, ausgenommen in Notfällen.\end{tabular} \\ \hline
				\begin{tabular}[c]{@{}l@{}}Geplante Pausen während der Arbeitszeit (z.B. Mittag)\\ gelten als Arbeitszeit und sind daher zu bezahlen.\end{tabular}  & Falsch  & \begin{tabular}[c]{@{}l@{}}sind keine Arbeitszeit und müssen nicht\\ bezahlt werden\end{tabular}                          \\ \hline
				\begin{tabular}[c]{@{}l@{}}Nach längstens 7 Stunden Arbeit gebührt dem Arbeitnehmer\\ eine mindestens halbstündige Pause.\end{tabular}              & Falsch  & nach längstens 6 Stunden                                                                                                  \\ \hline
				\begin{tabular}[c]{@{}l@{}}Es dürfen täglich maximal 12 Stunden gearbeitet werden,\\ wenn keine Arbeitsbereitschaft vorliegt\end{tabular}           & Richtig &                                                                                                                           \\ \hline
			\end{tabular}
		\end{table}
	\newpage
	\subsection{Ü3.18}
		\textbf{Welche Teilbereiche umfasst die Sozialversicherung und durch welche Institutionen erfolgt die Verwaltung?}
		Im engeren Sinn: Kranken-, Unfall- und Pensionsversicherung. Die Verwaltung erfolgt in Form der Selbstverwaltung durch
		die verschiedenen Sozialversicherungsträger. Im weiteren Sinn zählt dazu auch die Arbeitslosenversicherung – die Verwaltung
		erfolgt durch den Bund.
	
	\subsection{Ü3.19}
		\textbf{Wodurch unterscheidet sich die Selbstverwaltung von der staatlichen Verwaltung?}
		Die Selbstverwaltung ist kostengünstiger, demokratischer und unbürokratischer. Sie entlastet zudem die staatliche Verwaltung.
		Seit der zweiten Hälfte des 19. Jahrhunderts wird die Sozialversicherung nach dem Prinzip der Selbstverwaltung
		organisiert. Die Geschäfte werden von Vertretern der Versicherten und der Dienstgeber geführt.
	
	\subsection{Ü3.20}
		\textbf{Erklären Sie die beiden grundsätzlichen Verfahren zur Finanzierung eines Sozialversicherungssystems.} \newline
		\textbf{Umlageverfahren:} \newline
		Die Wirtschaftskraft der „aktiv“ Erwerbstätigen finanziert die Leistungen der „passiv“ Erwerbstätigen (Pensionisten). Die eingehobenen Beiträge der Arbeitnehmer werden daher nicht angespart, sondern gleich wieder z. B. für Pensionen ausgegeben. Dieses System wird auch als „Generationenvertrag“ bezeichnet. Probleme können dann auftauchen, wenn die Zahl der Beitragszahler sinkt und die Zahl der Leistungsempfänger steigt (siehe: Bevölkerungsprognose). \newline
		\textbf{Kapitaldeckungsverfahren:} \newline
		Prinzipiell spart hier jeder selbst für seine Altersabsicherung (und die seiner Familie) an.
		Ein Problem stellt die Veranlagung des Vermögens dar, da Inflation oder Wechselkursschwankungen den Ertrag starkverringern können. Durch persönliche oder wirtschaftliche Krisen kann es passieren, dass nicht ausreichende Rücklagen gebildet werden können.
		
	\subsection{Ü3.21}
		 \textbf{Sind folgende Aussagen richtig oder falsch?} \newline
		 Wer höhere Sozialversicherungsbeiträge bezahlt, erhält bessere Leistungen
		 (z.B. Medikamente, ärztliche Hilfe).  \textbf{FALSCH}
		 
		 Kinder sind bei ihren Eltern mitversichert.  \textbf{FALSCH}

	\subsection{Ü3.22}
		\textbf{Welche Leistungen kann man aus der gesetzlichen Krankenversicherung beziehen? }
		§ 117. Als Leistungen der Krankenversicherung werden nach Maßgabe der Bestimmungen dieses Bundesgesetzes gewährt:
		\begin{itemize}
			\item 	1. aus dem Versicherungsfall der Krankheit: Krankenbehandlung (§§ 133 bis 137);
			\item 	2. aus dem Versicherungsfall der Arbeitsunfähigkeit infolge Krankheit: Krankengeld (§§ 138 bis 143);
			\item 	3. aus dem Versicherungsfall der Mutterschaft:
				\subitem 	a) Hebammenbeistand, erforderlichenfalls ärztlicher Beistand (§ 159);
				\subitem	b) Heilmittel und Heilbehelfe (§ 160);
				\subitem	c) Pflege in einer Krankenanstalt (auch in einem Entbindungsheim) (§ 161);
				\subitem	d) Wochengeld, Stillgeld (§§ 162 und 163);
				\subitem	e) Entbindungsbeitrag (§ 164);
			\item	4. aus dem Versicherungsfall des Todes: Sterbegeld (§§ 169 bis 171).
			§ 118. (1) An Stelle der Krankenbehandlung und gegebenenfalls des Krankengeldes treten nach Maßgabe der §§ 144 bis
			150 Anstaltspflege oder nach Maßgabe des § 151 Hauspflege und gegebenenfalls Familien(Tag)geld (§ 152).
				 \subitem	(2) Zahnbehandlung und Zahnersatz sowie Hilfe bei körperlichen Gebrechen werden nach Maßgabe der Bestimmungen
				 der §§ 153 und 154 gewährt.
				 \subitem	(3) Überdies können Leistungen der erweiterten Heilfürsorge (§ 155) und Leistungen zur Verhütung des Eintrittes
				 und der Verbreitung von Krankheiten (§ 156) gewährt werden.
				 § 119. Die Leistungen der Krankenversicherung werden auch gewährt, wenn es sich um die Folgen eines Arbeitsunfalles (§§ 175 und 176) oder um eine Berufskrankheit (§ 177) handelt.
		\end{itemize}
		
	\subsection{Ü3.23}
		\textbf{Kann man im Ausland Leistungen aus der gesetzlichen Krankenversicherung beziehen?} \\
		Ja, es ist eine kostenlose Behandlung bei Vertragsärzten und öffentlichen Spitäler im gesamten EU-Raum, EWR-Raum und der Schweiz möglich. \\
		\textbf{Wodurch kann man diese Leistungen in Anspruch nehmen?} \\
		Mithilfe der EKVK (europäische Krankenversicherungskarte) 
	
	\subsection{Ü3.24}
		\textbf{Für welche Unfälle kommt die gesetzliche Unfallversicherung auf?} \\
		für Arbeitsunfälle, das sind Unfälle, die sich im örtlichen, zeitlichen und ursächlichen Zusammenhang mit der Beschäftigung,
		die die Versicherung begründet, ereignen 
		
	\subsection{Ü3.25}
		\textbf{Für welchen Personenkreis gilt das Allgemeine Pensionsgesetz?} \\
		Im Jahr 2004 wurde das Pensionsrecht durch das PensionsharmonisierungsG (BGBL I 2004/142) ein weiteres Mal geändert und das Allgemeine Pensionsgesetz (APG) erlassen:
		
		\begin{itemize}
			\item Für Personen, die ab dem 1.1.2005 erstmals in der gesetzlichen Pensionsversicherung versichert sind, gilt vorrangig das APG.
			\item Für Personen, die bereits vor dem 1.1.2005 in der gesetzlichen Pensionsversicherung versichert waren, gilt ein Mix aus
			Alt- und Neurecht (Parallelrechnung).
			\item Für die am 1.1.2005 bereits über 50-Jährigen gilt weiterhin das Leistungsrecht in der Pensionsversicherung nach dem
			ASVG, GSVG, FSVG und BSVG, jedoch mit zwei Ausnahmen: Diese Personen können auch die Korridorpension (ab
			dem 62. Lebensjahr) sowie die Schwerarbeiterpension in Anspruch nehmen.
		\end{itemize}
		
	\subsection{Ü3.26}
		\textbf{Nennen und erklären Sie die Leistungen aus der Arbeitslosenversicherung.} \\
		Arbeitslosengeld:\\
		erhalten jene Personen, die arbeitsfähig, arbeitswillig und verfügbar sind sowie die Anwartschaft
		erfüllt haben. \\
		Notstandshilfe: \\
		kann man beantragen, wenn der Anspruch auf Arbeitslosengeld erschöpft ist, aber eine Notlage vorliegt.
		Für die Berechnung der Höhe der Notstandshilfe ist der Grundbetrag des Arbeitslosengeldes, das anrechenbare
		Einkommen und die Familiengröße ausschlaggebend. Die Höhe beträgt 92\% bis 95\% des Grundbetrages des Arbeitslosengeldes.
		Das Partnereinkommen wird hier im Gegensatz zum Arbeitslosengeld jedoch angerechnet, sofern es nicht
		geringfügig ist.
	
	\subsection{Ü3.27}
		\textbf{Wer finanziert die Familienbeihilfe und wie lange kann diese bezogen werden? }\\
		Die Finanzierung erfolgt aus dem FLAF (Familienlastenausgleichsfonds). Der FLAF wird hauptsächlich durch den „Dienstgeberbeitrag
		zum FLAF“ gespeist. \\
		Dauer des Anspruchs:
		\begin{itemize}
			\item grundsätzlich bis zum vollendeten 18. Lebensjahr 
			\item bei Schülern und Studenten in Ausbildung bis max. zum 24. Lebensjahr (bzw. in Ausnahmefällen bis zum 25. Lebensjahr).
			Es müssen jeweils Nachweise über den Schulerfolg erbracht werden.
		\end{itemize}
		
	\subsection{Ü3.28}
		\textbf{Welche Voraussetzungen muss man erfüllen, um Kinderbetreuungsgeld beziehen zu können? }
		\begin{itemize}
			\item 	Anspruch auf Familienbeihilfe
			\item	gemeinsamer Haushalt mit dem Kind
			\item 	Durchführung der Mutter-Kind-Pass-Untersuchungen
			\item 	Beachtung der Zuverdienstgrenze (die Einkünfte dürfen im Kalenderjahr € 16.200,– nicht übersteigen.)
		\end{itemize}
		
	\subsection{Ü3.29}
		\textbf{Wer kann Pflegegeld beziehen?} \\
			Jede Person, die einen ständigen Pflegebedarf wegen einer körperlichen, geistigen oder psychischen Behinderung hat,
			wobei der monatliche Pflegebedarf mehr als 50 Stunden betragen muss und dieser Zustand wahrscheinlich mehr als
			6 Monate andauern wird. \\
		\textbf{Hängt die Höhe des Pflegegeldes vom Einkommen ab? (Begründung) } \\
			Nein, die Höhe des Pflegegeldes stellt nur auf das Ausmaß des Pflegebedarfs ab. Außerdem ist es nur als pauschalierter
			Beitrag zu den tatsächlichen Pflegekosten zu sehen, da die tatsächlichen Pflegkosten wesentlich höher sind.
			
	\subsection{W3.17}
		Der Angestellte Karl Gruber wurde von seinem Arbeitgeber gekündigt und meldet sich nun beim AMS als
		arbeitslos.
		\begin{itemize}
			\item  Wovon hängt es ab, ob Herr Gruber überhaupt Arbeitslosengeld bekommt?
			ob er arbeitsfähig, arbeitswillig und verfügbar ist.
			\item Wie lange kann Herr Gruber Arbeitslosengeld beziehen?
			Dies hängt davon ab, wie lange er vor der Kündigung innerhalb eines bestimmten Zeitraumes arbeitslosenversicherungspflichtig
			beschäftigt war (= Anwartschaft).
			\item Was passiert, wenn sein Anspruch auf Arbeitslosengeld ausgeschöpft ist und er noch immer keine Arbeit hat?
			Falls eine Notlage vorliegt, kann er Notstandshilfe beantragen.
		\end{itemize}
	
	\subsection{W3.18}
		Während Herr Gruber arbeitslos ist, bekommt er furchtbare Zahnschmerzen, sodass er einen Zahnarzt aufsuchen
		muss.
		\begin{itemize}
			\item Wer bezahlt die Zahnarztrechnung?
			Da auch Arbeitslose krankenversichert sind, bezahlt dies die Krankenkasse.
			\item Was benötigt der Zahnarzt von ihm, um ihn zu behandeln?
			die E-Card 
		\end{itemize}	
		
	\subsection{W3.19}
		 Diskutieren Sie in Gruppen über die Vor- und Nachteile der verschiedenen Finanzierungsverfahren des Sozialversicherungssystems.
		 siehe dazu: Informationen über Umlage bzw. Kapitaldeckungsverfahren 
	
	\subsection{W3.20}
		\textbf{Überlegen Sie, ob es sinnvoll ist, dass z.B. die Sachleistungen aus der Krankenversicherung nicht prämienabhängig
			sind} \\
		 Ja, es würde sonst eine Zweiklassengesellschaft bzw. -medizin entstehen: Besserverdienende würden
		 Zugang zu Medikamenten, Heilbehelfen, ärztlicher Versorgung erhalten und Menschen mit geringem bzw. gar keinem
		 Einkommen wären davon ausgeschlossen. Das Solidaritätsprinzip der Sozialversicherung gewährleistet daher die Gleichbehandlung
		 von finanziell schlecht gestellten Menschen.
	
	\subsection{W3.21}
	
		\textbf{Maria Müller ist schwanger. Sie studiert im 3. Semester BWL und wird von ihren Eltern finanziert. Maria erkundigt
			sich nun nach verschiedenen Unterstützungsmöglichkeiten, da ihre Eltern andeuten, sie in Zukunft nicht mehr zu
			unterstützen.} Beraten Sie die Studentin, wo und in welcher Höhe sie Unterstützungen beantragen kann.
		\begin{itemize}
			\item Maria Müller kann Kinderbetreuungsgeld beantragen und sich für eines der 3 Modelle entscheiden. Diese Leistung ist
			nicht von einer Beschäftigung abhängig.
			\item Der Vater des Kindes kann zur Unterhaltszahlung herangezogen werden.
			\item Studierende mit Kind haben auch Anspruch auf die Höchststudienbeihilfe von € 606,– monatlich zuzüglich eines Zuschusses
			von € 44,– pro Monat.
		\end{itemize}
		
	\subsection{W3.22}
	
		\textbf{Martin Ott leidet an Altersdemenz und ist nicht mehr in der Lage, seinen Haushalt alleine zu bewältigen.
			Er braucht außerdem Hilfe bei der täglichen Körperpflege sowie beim Essen. }\\
		Suchen Sie im Internet (www.help.gv.at) nach Möglichkeiten der Betreuung und finanziellen Hilfe für Herrn Ott.
		Diskutieren Sie auch Schwierigkeiten bei der Finanzierung der Pflege allgemein. \\
		
		\begin{itemize}
			\item Man muss grundsätzlich überlegen, ob Herr Ott zu Hause versorgt werden kann oder eine Einweisung in ein Altenund
			Pflegeheim erforderlich ist.
			\item Zur finanziellen Unterstützung wurde das Pflegegeld geschaffen, das es Personen wie Herrn Ott ermöglichen würde,
			beim Vorhandensein einer entsprechender Betreuungskraft solange als möglich zu Hause betreut zu werden. Anträge
			sind im Internet abrufbar.
		\end{itemize}

		
\section{Sozialrecht}
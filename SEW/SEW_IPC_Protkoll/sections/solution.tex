%!TEX root=../document.tex

\section{Alogrithmus}
	\begin{lstlisting}[style=Python, caption= Simpler Algorithmus des Selbstdurchlauf]
	def doEverythingForMeSimple():
	i = 0
	j = 0
	fortfahren = True
	while(fortfahren):
	time.sleep(0.25)
	#ueberpruefen ob Lake am naechsten Feld
	if (len(fielddata)==18):
	print(fielddata[10])
	if (fielddata[10]=='L' and j!=4):
	goUp()
	j += 1
	i -= 1
	else:
	# Jeden Durchlauf eins nach unten
	if (i >= 9):
	if (fielddata[14] == 'L'):
	goRight()
	#i=-1
	else:
	goDown()
	i = 0
	elif (j >= 4):
	# wenn ausgewichen nach oben, dann wieder nach unten gehen
	goDown()
	i -= 1
	j = 0
	else:
	goRight()
	elif (len(fielddata)==50):
	print(fielddata[26])
	if (fielddata[26] == 'L' and j!=4):
	goUp()
	j += 1
	i -= 1
	else:
	# Jeden Durchlauf eins nach unten
	if (i >= 9):
	if (fielddata[34] == 'L'):
	goRight()
	#i = -1
	else:
	goDown()
	i = 0
	elif (j >= 4):
	# wenn ausgewichen nach oben, dann wieder nach unten gehen
	goDown()
	i -= 1
	j = 0
	else:
	goRight()
	elif (len(fielddata)==98):
	print(fielddata[50])
	if (fielddata[50] == 'L' and j!=4):
	goUp()
	j += 1
	i -= 1
	else:
	# Jeden Durchlauf eins nach unten
	if (i >= 9):
	if (fielddata[62] == 'L'):
	goRight()
	#i = -1
	else:
	goDown()
	i = 0
	elif (j >= 4):
	# wenn ausgewichen nach oben, dann wieder nach unten gehen
	goDown()
	i -= 1
	j = 0
	else:
	goRight()
	else:
	pass
	i += 1
	if (j>0):
	j += 1
	print("i: " + str(i))
	print("j: " + str(j))
	rec_fields(clientsocket)
	if not fielddata:
	fortfahren = False
	clientsocket.close()
	\end{lstlisting}
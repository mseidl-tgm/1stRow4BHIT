%!TEX root=../document.tex

\section{Ergebnisse}
\label{sec:Ergebnisse}
Aufgrund der Aufgabenstellung wurden folgende Funktionalitäten bei GlusterFS und bei OriFS konfiguriert und getestet.
\begin{itemize}
	\item replicate – Erstellung einer replizierten (verteilten) Speicherschnittstelle
	\item snapshot – Anlegen einer Sicherung
	\item newfs – Aufsetzen/Konfiguriation eines Dateisystems
	\item pull – Aktualisieren der lokalen Speicherständen mit Master-Entität
	\item checkout – Zurücksetzen auf früheren Zustand
	\item log – Auslesen von Informationen über bisherigen Zustände des Dateisystems
\end{itemize}

\section{GlusterFS}

\subsection{Installation}
	Um die Installation von GlusterFS durchzuführen müssen zwei virtuelle Maschinen aufgesetzt werden. Es gibt zwei Möglichkeiten dies zu tun, entweder über das Aufsetzen von zwei Systemen unter VirtualBox, oder die schnellere Variante unter Vagrant.
	\begin{lstlisting}[style=Java, caption=Vagrant Setup]
		vagrant init ubuntu/trusty64; vagrant up --provider virtualbox
	\end{lstlisting}
	Damit die Maschinen Ihren Speicher teilen können muss zunächst ein neues Netzwerk hinzugefügt werden, welches einem Host-Only entspricht. Zusätzlich muss noch eine Partition auf beiden Systemen erstellt werden, damit darin der gemeinsame Speicherbereich liegt. Bei beiden VMs ist die Partition, welche mittels fdisk erstellt wurde ein Gigabyte groß.\\
	All diese Einstellungen müssen im Vagrantfile eingestellt werden. \\
	Damit die gerade eben erstellten Festplatten verwendet werden können, müssen diese noch mit einem Dateisystem beschrieben werden. XFS wird als Dateisystem gewählt, da es alle Änderungen speichert und somit eine spätere Rekonstruierung möglich ist. Im nächsten Listing werden die Befehle für die Vorbereitung der Fesplatten vorgezeigt.
	\begin{lstlisting}[style=Java, caption=Vorbereitung der Festplatten]
		mkfs.xfs -i size=1024 /dev/sdb1
		mkdir -p /data/brick1
		echo '/dev/sdb1 /data/brick1 xfs defaults 1 2' >> /etc/fstab
		mount -a && mount
	\end{lstlisting}
	Um die Verbindung zwischen den beiden Maschinen über GlusterFS möglich zu machen, muss bei beiden VMs jeweils der Hostname der anderen Maschine in die /etc/hosts- Datei eingetragen werden. Jetzt muss zwischen den Maschinen ein trusted pool aufgebaut werden, welcher aus mehreren Maschinen bestehen kann. In diesem Beispiel sind aber nur die beiden Testmaschinen inkludiert. Damit so ein pool aufgesetzt werden kann muss ein Package mit dem Namen glusterfs-server installiert werden. Mittels eines probing zwischen den Maschinen wird so ein Pool aufgebaut.\\
	Jetzt müssen noch Ordner für den gemeinsamen Speicherbereich angelegt werden. Dies geschieht über gluster volume create, wie im folgenden Listing vorgeführt wird.
	\begin{lstlisting}[style=Java, caption=Gluster Volume]
		gluster volume create shared replica <number of bricks/servers> seidl_test1:shared_folder seidl_test2:shared_folder gluster volume start shared_folder
	\end{lstlisting}
	Mit der fertigen Installation sind bereits 3 der geforderten Anforderungen erfüllt. Durch das Anlegen von Dateien auf einer Maschine in dem geteilten Ordner, ist es jetzt möglich zu testen ob das ganze System funktioniert.
	\begin{lstlisting}[style=Java, caption=Erstellen einer Datei im Shared folder]
		touch testdatei.txt
		nano testdatei.txt
	\end{lstlisting}	
	
\section{OriFS}
\subsection{Installation}
	Die Installation erfolgt ebenfalls über zwei virtuelle Maschinen, welche mit Vagrant aufgesetzt und konfiguriert werden. Da OriFs nicht im offiziellen apt Repository vorhanden ist, muss es manuell aus einem anderen Repository hinzugefügt werden. Hierbei stellen die Entwickler ein Package in einem third-party Packagerepository (PPA)
	\begin{lstlisting}[style=Java, caption=Erstellen einer Datei im Shared folder]
		sudo add-apt-repository ppa:ezyang/ppa
		sudo apt update
		sudo apt install ori 
	\end{lstlisting}
	Die Dateiübertragung bei OriFS funktioniert über SSH und somit muss zwischen den beiden Hosts ein SSH-Zugriff ermöglicht werden. Hierbei muss ein gemeinsamer SSH-Key erstellt werden und bei den jeweiligen Hosts in die .ssh/authorized\_keys eingetragen werden.
	
\subsection{Konfiguration und Funktionsweise}
	OriFS erstellt Daten nicht auf Partitionen sondern in Repositories, diese müssen dann bei dem System gemountet werden. Das darauffolgende Listing beschreibt die Erstellung eines solchen Repositories und das mounten über die benötigten Befehle.
	\begin{lstlisting}[style=Java, caption=Konfiguration eines OriFS Repositories]
		ori newfs testRepo
		mkdir testRepo
		orifs testRepo
	\end{lstlisting}
	Jetzt ist auf einer der beiden Testmaschinen ein Repository erstellt worden. Damit aber die zweite virtuelle Maschine ebenfalls auf diesen Datenbereich zugreifen kann, hat OriFS eine ähnliche Funktionsweise wie Git. Um eine lokale Änderung im Repository zu übernehmen muss ein sogenanter snapshot gemacht werden, welcher wie ein commit von Git funktioniert. Auf einem anderen System kann jetzt mittels dem Befehl pull die gesammelten snapshots lokal auf die Maschine geladen werden. Bei gleichzeitigen Änderungen im Repository von verschiedenen Standorten, muss ein merch ausgeführt werden um die unterschiedlichen snapshots zusammen zu führen. Im nächsten Listing werden die gerade genannten Befehle dargestellt.
	\begin{lstlisting}[style=Java, caption=OriFS Befehle]
	ori replicate seidl@192.168.66.12:testRepo
	mkdir testRepo	
	orifs testRepo
	\end{lstlisting}	
	\begin{lstlisting}[style=Java, caption=OriFS Befehl pull und snapshot]
	ori pull
	Pulled up to edaf575111225a8a120cb98081a649a82a8915a7f01cfb04848f82456b5012f5
	
	ori checkout edaf575111225a8a120cb98081a649a82a8915a7f01cfb04848f82456b5012f5
	Checkout success!
	
	ori snapshot
	Committed edaf575111225a8a120cb98081a649a82a8915a7f01cfb04848f82456b5012f5
	\end{lstlisting}	

\subsection{Synchronisierung}
	Mit den oben beschriebenen Befehlen erhält man keine automatische Synchronisierung zwischen den Maschinen, sondern es muss jede Änderung manuell von der anderen Maschine geholt werden. OriFS bietet hierfür den orisync-Dienst an, welcher das snapshoten und pullen automatisiert, da er Änderungen von selber erkennt. \\
	
	Theoretisch sollte dies einwandfrei nach Aktivierung dieses Dienstes funktionieren, jedoch hat es bei mir nach mehreren Versuchen nicht funktioniert. Das erstellen eines Clusters mittels dem Begfehl orisync init, sollte die Maschinen zusammenführen, doch hierbei traten immer wieder Fehler auf.

\subsection{Logging}
	In OriFS ist ein Einblick in log-Dateien sehr einfach gestaltet und funktioniert über den Befehl ori log. Dieser gibt einen Einblick in die Meta-Daten eines Repositorys.
	\begin{lstlisting}[style=Java, caption=ori log]
	Name File				 System ID
	testRepo				 be52494d-78f7-423c-9347-083e6299a1f8
	seidl@vagrant-ubuntu-trusty-64:~/testRepo$ ori log
	Commit: 	134608697e8308ca056c87dda42ba7e8c2cc1ef3749a57384201220cd75a6921
	Parents: 	edaf575111225a8a120cb98081a649a82a8915a7f01cfb04848f82456b5012f5
	Tree: 		850eb6fab4785c419a30b6619d78f590d80e46a85e348f4cbfb241606fc2436e
	Author: Date: Wed Apr 05 17:12:33 2017
	Created snapshot 'firstSnap'
	Commit: 	edaf575111225a8a120cb98081a649a82a8915a7f01cfb04848f82456b5012f5
	Parents: 
	Tree: 		52869c32902bf5245a41a0028b48f7768d1900d403ec12f8302c56e91d50539f
	Author: 
	Date: Wed Apr 05 16:46:04 2017
	
	No message.
	\end{lstlisting}
	
\section{Fazit - Gegenüberstellung}
	Aufgrund der Probleme bei der Installation und Konfiguration von OriFS war es mir relative schwer einen sinnvollen Zusmamenhang zwischen den beiden verteilten Dateisystemen zu finden. Abgesehen von der Funktionsweise unterscheiden sich die beiden DFS lediglich im Einsatzbereich. OriFS zum Beispiel wäre keine gute Lösung für ein verteiltes Dateisystem in einem Unternehmen, da es seit 2014 keine weiteren Updates gegeben hat und die Dokumentation nur spärlich vorhanden ist.
	GlusterFS dagegen ist nach einigen problemlosen Schritten voll einsatzfähig und ist deswegen auch sehr weit verbreitet. Der native Support von verschiedenen Linux-Distiributionen ist ebenfalls ein weiterer Punkt, welcher GlusterFS über OriFS stellt. Der einzige Nachteil den Schulkollegen im Laufe der Durchführung der Laborübung entdeckt haben, ist die Vernachlässigung von Linux-Paketen, welche auf nicht RedHat-basierten Distributionen vorkommt.
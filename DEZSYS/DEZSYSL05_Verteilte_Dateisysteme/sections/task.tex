%!TEX root=../document.tex

\section{Installation und Implementierung}
\begin{center}
	Ori is a distributed file system built for offline operation and empowers the user with
	control over synchronization operations and conflict resolution. We provide history
	through light weight snapshots and allow users to verify the history has not been
	tampered with. Through the use of replication instances can be resilient and recover
	damaged data from other nodes.
\end{center}
Installieren Sie Ori und testen Sie die oben beschriebenen Eckpunkte dieses verteilten Dateisystems
(DFS). Verwenden Sie dabei auf jeden Fall alle Funktionalitäten der API von Ori um die
Einsatzmöglichkeiten auszuschöpfen. Halten Sie sich dabei zuallererst an die Beispiele aus dem
Paper im Kapitel 2 [2]. Zeigen Sie mögliche Einsatzgebiete für Backups und Roadwarriors (z.B.
Laptopbenutzer möchte Daten mit zwei oder mehreren Servern synchronisieren). Führen Sie auch
die mitgelieferten Tests aus und kontrollieren Sie deren Ausgaben (Hilfestellung durch Wiki [3]).

\subsection{Gegenüberstellung}
Wo gibt es Überschneidungen zu anderen Implementierungen von DFS? Verwenden Sie dabei
HDFS [4], GlusterFS [5] oder ein DFS Ihrer Wahl als Gegenspieler zu Ori. Um aussagekräftige
Vergleiche anstellen zu können, wäre es von Vorteil die anderen Systeme ebenfalls - zumindest
oberflächlich - zu testen. Dokumentieren Sie die Einsatzgebiete der DFS.


\subsection{Info}
Gruppengröße: 2 Mitglieder
Gesamtpunkte: 16
\begin{itemize}
	\item Installation und Testdurchlauf von OriFS: 3 Punkte
	\item Installation und Testdurchlauf eines anderen DFS: 3 Punkte
	\item Einsatz/Dokumentation der Ori API (Waehlen Sie 6 Funktionen aus folgender Liste: replicate, snapshot, checkout, graft, filelog, list, log, merge, newfs, pull, remote, removefs, show, status, tip, varlink): 4 Punkte
	\item Gegenüberstellung der beiden DFS (Allgemein, Staerken, Schwaechen): 4 Punkte
	\item Einsatzgebiete von DFS : 2 Punkte
\end{itemize}


\subsection{Quellen}
\begin{itemize}
	\item Ori File System, Stanford Website, online: http://ori.scs.stanford.edu/, visited: 2016-04-01
	\item Ori File System, Bitbucket Wiki, online: https://bitbucket.org/orifs/ori/wiki/Home, visited: 2016-04-01
	\item Ali José Mashtizadeh, Andrea Bittau, Yifeng Frang Huang, David Mazières. Replication, History, and Grafting in the Ori File System. In Proceedings of the 24th Symposium on Operating Systems Principles, November 2013. Paper.
	\item Apache Hadoop FileSystem, http://hadoop.apache.org/docs/current/hadoop-project-dist/hadoop-hdfs/HdfsUserGuide.html, visited: 2016-04-01
	\item GlusterFS, http://gluster.readthedocs.org/en/latest/, visited: 2016-04-01
\end{itemize}

\clearpage
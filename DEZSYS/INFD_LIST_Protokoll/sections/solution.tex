%!TEX root=../document.tex
\section{Aufgabenstellung}
Nachbau eines Garagentor mittles eine Induktionsplatte. Beim simulierten Aufgehen des Tores soll eine rote Lampe leuchten. Es gibt einen Motor mit zwei Eingängen (1/EIN und 0/Aus) und (DIR 1/nach oben und 0/nach unten). Außerdem soll das Garagentor auch eine Lichtschranke besitzen welche \textbf{0 für unterbrochen} und \textbf{1 für nicht unterbrochen} sendet. Ein Endschalter sollte zur Sichherheit auch vorhanden sein (\textbf{1 Kontakt sonst 0})
\textbf{Der Ablauf sollte dann folgendermaßen Aussehen:}
\begin{itemize}
	\item Auto da -> Tor aufgehen
	\item Tor bewegt sich -> rot Lampe an
	\item Tor offen -> grüne Lampe
\end{itemize}}
	Inhalt...
\end{i}
\section{Ergebnisse}
\label{sec:Ergebnisse}
Es standen ein paar Beispielklassen zur Verfügung, welche von Moodle heruntergeladen wurden. Darin befanden sich zwei Klassen, eine davon beschrieb die den LDAP Connector (wird später noch im Protokoll erwähnt) und die andere erklärte ein paar Code-Snippets zur Verschlüsselung und Encoding. Unter anderem gab es schon zwei Utils-Methoden, welche das Encoding und Umwandeln deutlich erleichtert haben.




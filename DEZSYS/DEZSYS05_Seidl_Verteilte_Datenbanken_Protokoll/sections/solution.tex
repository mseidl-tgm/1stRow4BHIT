%!TEX root=../document.tex

\section{Ergebnisse}
\label{sec:Ergebnisse}

\subsection{Aufbau der Entwicklungsumgebung}

\subsubsection{Virtuelle Maschine}

Wie in der Aufgabenstellung vorgesehen, habe ich mir zwei virtuelle Maschinen in Virtual Box aufgesetzt. Das verwendete Betriebssystem ist hierbei \textit{ubuntu-server 14.04}, welches ich von der offiziellen Ubuntu-Seite heruntergeladen habe.
Da das Aufsetzen einer VM hier nicht in die Dokumentation gehört, springe ich direkt in ein fertiges Betriebssystem.

\subsubsection{Netzwerkkonfiguartion}

Damit die Kommunikation zwischen den beiden virtuellen Maschinen und dem Host gewährleistet wird, müssen die Netzwerkkarten der Server auf \textbf{Netzwerkbrücke/Networkbridge} umgestellt werden. Des Weiteren muss der \textbf{``Promiscuous-Modus''} auf \textbf{``erlauben für alle VMs und den Host''} gestellt werden, damit eine fehlerfreie Übertragung möglich ist.

\subsubsection{PostgreSQL Installation}
Auf beiden VMs wird PostgreSQL aufgesetzt, um die Datenbank zu verteilen. Zunächst wird das benötigte Package in einem Packet-Manager heruntergeladen und installiert. Bei Ubuntu handelt es sich hierbei um \textit{aptitude}.
\begin{lstlisting}[language=bash,caption={PostgreSQL Installation}]
sudp apt-get install postgreSQL
\end{lstlisting}

\subsubsection{Datenbankkonfiguration - User, Rechte}
Bei der Erstkonfiguration von PostgreSQL wird ein default-User mit dem Namen \textit{postgres} angelegt. Mit diesem Benutzer stellt man eine Verbindung zum PostgreSQL-Server her. Danach führt man folgende Schritte durch, um sich den User für diese Aufgabe mit den entsprechenden Rechten anzulegen.

\begin{lstlisting}[language=bash,caption={User, Rechte anlegen}]

postgres=# CREATE ROLE ds2 WITH superuser login;
postgres=# ALTER ROLE ds2 PASSWORD 'ds2';

seidl@seidl:~$ psql -U ds2 -d ds2
\end{lstlisting}

\newpage
Als nächstes muss noch das Konfigurations-File so bearbeitet werden, dass die Datenbank Zugriffe von allen Seiten annimmt.
Hierbei werden die Files \textbf{/etc/postgresql/9.5/main/postgresql.conf} und \textbf{pg-hba.conf} bearbeitet.
Im folgenden Listing werden die einzutragenden Kommandos dargestellt:
\begin{lstlisting}[language=bash,caption={PostgreSQL Konfiguration}]
host	ds2		ds2		samenet		trust

listen_addresses= '*'
\end{lstlisting}

\subsection{PostgreSQL Remote-Server Konfiguration}
Jetzt muss nur noch der Remote-Server richtig den Local-Server als \textbf{foreign-server} eingetragen werden. Dies erfolgt mit folgenden Kommandos in der psql-Konsole:
\begin{lstlisting}[language=bash,caption={Remote-Server Konfiguration}]
	CREATE EXTENSION postgres_fdw;
	CREATE SERVER foreign_server
	FOREIGN DATA WRAPPER postgres_fdw
	OPTIONS
		(host '10.0.106.18', port '5432', dbname 'ds2');
	CREATE USER MAPPING FOR ds2
	SERVER foreign_server
	OPTIONS(user 'ds2', password 'ds2');
\end{lstlisting}

\subsection{Horizontale Fragmentierung}

Für die horizontale Fragmentierung habe ich mir bei der Tabelle \textbf{customers} überlegt, mit folgenden Trennungen:
\begin{itemize}
	\item gender = 'F' - Bedeutet alle Einträge welche dem weiblichen Geschlecht angehören
	\item age < 21 - und jünger als 21 sind.
\end{itemize}

Damit die Fragmentierung möglich ist, muss man eine \textbf{FOREIGN TABLE} am Remote-Server erstellen.
Diese Tabellen holen sich dann die Daten von dem Local-Server und trägt diese in die Tabelle ein.
Mit einem simplen SELECT-Statement kann man das ganze überprüfen:
\begin{lstlisting}[language=bash,caption={Horizontale Fragmentierung}]
SELECT COUNT(*) FROM horizontal.remote_customers_f_under_21;
\end{lstlisting}
\newpage
\subsection{Vertikale Fragmentierung}
 Für die vertikale Fragmentierung erfolgt ebenfalls eine Trennung der Tabelle \textbf{customers},
 aber diesmal werden die Spalten getrennt. Die Tabelle wird unterteilt in \textbf{payment-info} und
 \textbf{non-payment-info}. Zu beachten bei der vertikalen Fragmentierung ist das Eintragen der PRIMARY KEYS in \textbf{beide Tabellen} (sowhol FOREIGN als auch local).
 
 \begin{lstlisting}[language=bash,caption={Vertikale Fragmentierung}]
 INSERT INTO vertical.customers_payment_info
 SELECT customerid, creditcardtype, creditcard, creditcardexpiration, age, income FROM customers;
 \end{lstlisting}
 
 \subsection{Kombinierte Fragmentierung}
 Die kombinierte Fragmentierung ist die Mischung einer horizontalen, sowie einer vertikalen Fragmentierung.
 Ich habe mir hierfür folgende Unterteilung überlegt:
 
 \begin{itemize}
 	\item Vertikale Fragmentierung
	 	\subitem Vertrauliche Daten z.B.: Kreditkarte, Passwort
	\item Horizontale Fragmentierung
		\subitem Alle Kunden aus den US werden getrennt von den Kunden außerhalb.
 \end{itemize}
 
 \section{SELECT-Resultate}
 
 \textbf{Horizontale Fragmentierung}
 \begin{lstlisting}[language=bash,caption={SELECT-horizontal}]
	SELECT COUNT(*) FROM horizontal.customers_f_under_21;
	count
	------
	429
	(1 row)
	
	SELECT COUNT(*) FROM horizontal.remote.customners_f_under_21;
	count
	------
	429
	(1 row)
 \end{lstlisting}
 
 \textbf{kombinierte Fragmentierung}
  \begin{lstlisting}[language=bash,caption={SELECT-combined}]
	 SELECT * FROM combined.customers_us_name_phone
	 NATURAL JOIN
	 combined.remote_customer_non_us_name_phone
	 UNION
	 SELECT * FROM combined.customer_non_us_name_phone
	 NATURAL JOIN
	 combined.remote_customer_us_name_phone;
  \end{lstlisting}

Als Resultat bekommt man wieder eine vollständige \textbf{customer}-Tabelle.
 

\subsection{Zeitaufwand}
\renewcommand{\arraystretch}{1.5}
\begin{table}[!h]
	\center
	\begin{tabular}{ | @{\hspace{3mm}} c @{\hspace{3mm}} | @{\hspace{3mm}} l @{\hspace{3mm}} | }
		\hline Zeitaufwand & Task\\ \hline\hline
		\textbf{4h} & Schule Labor, Fragmentierungsmethoden ueberlegt\\ \hline
		\textbf{3h} & Entwicklungsumgebung und Serverkonfiguration + PostgreSQL\\ \hline
		\textbf{5h} & Umsetzung der Fragmentierungsvarianten\\\hline
	\end{tabular}
	\caption{Zeitaufzeichnung}
	\label{methoden}
\end{table}


